\documentclass[]{article}
\usepackage{amsmath}


\title{Uniform Circular Motion and Rotational Dynamic}
\author{}
\begin{document}
    \maketitle
    \section{Uniform Circular Motion}\label{sec:uniform-circular-motion}
        Uniform means that the object's velocity have the same magnitude all
        the time. Assuming an object has velocity $v$ when traveling around
        the center with radius $r$. The amount of time($t$) it will take to
        complete one cycle will be the total distance divide by the magnitude of
        the velocity.
            \[t = \frac{2\pi r}{v} \]
        At the same time, it can also be looked as the object has completed $2\pi$ radians of angle,
        and it takes $t$ amount of time to complete, so that the angular
        velocity($\omega$) in $\frac{rad}{sec}$ has formula:
            \[\omega = \frac{2\pi}{t} = \frac{v}{r}\]
        In other form:
            \[v = \omega r\]
        It can also be thought as the vector $\vec{r}$, like a clock's hand, is
        sweeping across the circle with angular velocity $\omega$ so that the
        person on the end point of the vector or the clock's hand is
        experiencing a velocity $v$.
    \section{Centripital Acceleration}\label{sec:centripital-acceleration}
        Because the vector $\vec{v}$ is always changing direction, there
    must be an acceleration on the object. When an object completed a cycle,
    the vector $\vec{v}$ mus also completed one cycle as it is the same as
    where it started after $t$ seconds. Similar to the derivation of $v$, the
    vector $\vec{v}$ can be thought as a clock's hand, sweeping across the circle with
    angular velocity $\omega$. So that the person on the end point of the
    vector or the clock's hand is experiancing, the velocity of the velocity
    vector $a$ aka Centripital acceleration.
    \[a = \omega v\]
    Or using that the circumference of the circle is equal in both expressions:
    \[a t  = \omega v t\]
    \[a = \omega v\]
    Since $v = \omega r $, the commonly usable formulas are:
        \begin{gather*}
            a = \omega ^2 r\\
            a = \frac{v^2}{r}\\
        \end{gather*}
    \section{Centripital Force}\label{sec:centripital-force}
    Centripital force can be in many different forms:
    \begin {enumerate}
        \item gravitational force such as the Sun and Earth, and the Earth
        and Moon.
        \item String tension swinging a rope in a circle.
        \item normal force in the space station is rotating, and the person
        inside of the car when car is turning.
        \item Spring force when a spring is rotating.
        \item static friction when the car is turing. It is static because no
        heat is generated as the result of turning the car.
    \end{enumerate}
    The centripital force is the net result of the other forces on the object,
    it alone is not a force. It is simply the force that keeps an
    object in circular motion.
    \[F_c = m a_c = m \omega ^2 r = m\frac{v^2}{r}\]

    \section{Work done by Centripital Force}
        When the object is has no tangential acceleration, at that short
        period of time, it can be thought that the object is in uniform
        circular motion.
        This is because the kinetic energy of the object at all points are:
            \[KE = \frac{1}{2} m v^2\]
        Since the kinetic energy is constant, the work done by the
        centripital force is Zero.
    \section{Tangential Acceleration}
        When the object is not in uniform circular motion, the object is
        experiencing a tangential acceleration. The tangential acceleration
        is the rate of change of the magnitude of the tangential velocity. The
        acceleration is in the direction of the velocity vector, and it is
        the result of the net force on the object. Since the object's
        tangential velocity is changing, its angular velocity is also
        changing with proportion to the tangential acceleration. As :
        \begin{gather*}
            \Delta \omega = \frac{\Delta v}{r}\\
            \frac{\Delta\omega}{\Delta t} = \frac{\Delta v}{\Delta t}
            \frac{1}{r} = \frac{a_t}{r}\\
            \alpha = \frac{\Delta \omega}{\Delta t} = \frac{a_t}{r}
        \end{gather*}
        Where $\alpha$ is the angular acceleration, the final equation can
    also be written as:
        \[a_t = r \alpha\]

    \section{Rotational Kinematics}
        Previously, we are able to derive the relationship between linear
        tems and angular terms. The relationship is:
        \begin{gather*}
            x = \theta r\\
            v = \omega r\\
            a_t = \alpha r\\
            a_c = \omega v\\
        \end{gather*}

        The rotational kinematics is similar to the linear kinematics, but
        instead of using $x$, $v$, and $a$, it uses $\theta$, $\omega$, and
        $\alpha$. The equations are:
        \begin{gather*}
            \theta = \theta_0 + \omega_0 t + \frac{1}{2} \alpha t^2\\
            \omega = \omega_0 + \alpha t\\
            \omega^2 = \omega_0^2 + 2\alpha(\theta - \theta_0)
        \end{gather*}
        The formulas that are valid for linear kinematics are also valid for
        rotational motion.
%    \sectoion{Simple Machine}
%        Simple machines are devices that can be used to multiply or change
%        the direction of a force. The six simple machines are:
%        \begin{enumerate}
%            \item Lever
%            \item Wheel and Axle
%            \item Pulley
%            \item Inclined Plane
%            \item Wedge
%            \item Screw
%        \end{enumerate}
%        The idea of a simple machine is that the work done by the machine is the
%        same as the work done on the machine.
%        \[W_{in} = W_{out}\]
%        The mechanical advantage of a
%        simple machine is the ratio of the force exerted by the machine to the
%        force applied to the machine. On a lever, the mechanical advantage is
%        the ratio of the length of the lever arm to the length of the force
%        arm.
    \section{Torque}
        In understanding the concept of torque, it is easier to gain an
        intuitive sense of it with the context of simple machines. A lever
        is can a good example.

        In analyzing the force amplified by a lever, the concept is :
        \[W_{in} = W_{out}\]
        We can start with one side of the lever
        \[F_{tan1} \cdot r_{1}\Delta \theta\]
        on the other side of the lever, the force is:
        \[F_{tan2}\cdot r{2}\Delta \theta\]
        If we disreguard the angular portion of the work done, the result
        will give us the torque of the lever in units of Jules.
        \[\tau = r F_{tan}\]
        The unit for torque is the same as work because it is from the work
        energy but the "distane" is incomplete because its distance is
        missing radians.
        Torque is the rotational equivalent of force. It is the force that
        causes an object to rotate. The formula for torque is:
        \[\tau = r F \sin(\theta)\]
        Where $\theta$ is the angle between the force and the lever arm. And
    the extra term of $\sin \theta$ is a projection of the force to the
    tangential portion since the centripital force doesn't do any work.
    \section{Moment of Inertia}
    When calculating the kinetic energy of a point mass in circular motion,
    assuming it is a point mass, the formula is:
        \[KE = \frac{1}{2}mv^2\]
    Since $v = \omega r$, the formula can be rewritten as:
        \[KE = \frac{1}{2}m(\omega r)^2 = \frac{1}{2}m r^2 \omega^2\]
    The term $mr^2$ is called the moment of inertia($I$) of the object which
    measures how object resists to rotate by the force $\alpha$. Different
    shapes of object will have a different moment of inertia, and rotating
    at a different point will also change the moment of inertia of the same
    object. Moment of inertia is just a new definition of mass, like how
    mass measures resistence an object is to move linearly.
    \section{Rotational Kinetic Energy}
    The formula for rotational kinetic energy derived from the last section is:
        \[KE = \frac{1}{2} I \omega^2\]
    The formula is similar to the linear kinetic energy, but instead of using
    mass, it uses moment of inertia.
    \section{Conservation of Angular Momentum}
    Angular momentum, similar to linear momentum, is defined as:
        \[L = \Sigma \tau \Delta t  = I \Sigma \alpha \Delta t = I\omega\]
    Different from consercation of energy that can be used in any movement of an object,
    conservation of angular momentum can only be used in a rotational
    ocntext and cannot be used with linear momentum.

\end{document}
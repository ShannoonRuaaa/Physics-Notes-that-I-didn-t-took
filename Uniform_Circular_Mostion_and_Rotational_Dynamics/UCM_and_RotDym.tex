\documentclass[]{article}
\usepackage{amsmath}


\title{Uniform Circular Motion and Rotational Dynamic}
\author{}
\begin{document}
    \maketitle
    \section{Uniform Circular Motion}\label{sec:uniform-circular-motion}
        Uniform means that the object's velocity have the same magnitude all
        the time. Assuming an object has velocity $v$ when traveling around
        the center with radius $r$. The amount of time($t$) it will take to
        complete one cycle will be the total distance divide by the magnitude of
        the velocity.
            \[t = \frac{2\pi r}{v} \]
        At the same time, it can also be looked as the object has completed $2\pi$ radians of angle,
        and it takes $t$ amount of time to complete, so that the angular
        velocity($\omega$) in $\frac{rad}{sec}$ has formula:
            \[\omega = \frac{2\pi}{t} = \frac{v}{r}\]
        In other form:
            \[v = \omega r\]
        It can also be thought as the vector $\vec{r}$, like a clock's hand, is
        sweeping across the circle with angular velocity $\omega$ so that the
        person on the end point of the vector or the clock's hand is
        experiencing a velocity $v$.
    \section{Centripital Acceleration}\label{sec:centripital-acceleration}
        Because the vector $\vec{v}$ is always changing direction, there
    must be an acceleration on the object. When an object completed a cycle,
    the vector $\vec{v}$ mus also completed one cycle as it is the same as
    where it started after $t$ seconds. Similar to the derivation of $v$, the
    vector $\vec{v}$ can be thought as a clock's hand, sweeping across the circle with
    angular velocity $\omega$. So that the person on the end point of the
    vector or the clock's hand is experiancing, the velocity of the velocity
    vector $a$ aka Centripital acceleration.
    \[a = \omega v\]
    Since $v = \omega r $, the commonly usable formulas are:
        \begin{gather*}
            a = \omega ^2 r\\
            a = \frac{v^2}{r}\\
        \end{gather*}
    \section{Tangential Acceleration}

\end{document}
\documentclass[]{article}
\usepackage{amsmath}


\title{Uniform Circular Motion and Rotational Dynamic}
\author{}
\begin{document}
    \maketitle
    \section{Uniform Circular Motion}\label{sec:uniform-circular-motion}
        Uniform means that the object's velocity have the same magnitude all
        the time. Assuming an object has velocity $v$ when traveling around
        the center with radius $r$. The amount of time($t$) it will take to
        complete one cycle will be the total distance divide by the magnitude of
        the velocity.
            \[t = \frac{2\pi r}{v} \]
        At the same time, it can also be looked as the object has completed $2\pi$ radians of angle,
        and it takes $t$ amount of time to complete, so that the angular
        velocity($\omega$) in $\frac{rad}{sec}$ has formula:
            \[\omega = \frac{2\pi}{t} = \frac{v}{r}\]
        In other form:
            \[v = \omega r\]
        It can also be thought as the vector $\vec{r}$, like a clock's hand, is
        sweeping across the circle with angular velocity $\omega$ so that the
        person on the end point of the vector or the clock's hand is
        experiencing a velocity $v$.
    \section{Centripital Acceleration}\label{sec:centripital-acceleration}
        Because the vector $\vec{v}$ is always changing direction, there
    must be an acceleration on the object. When an object completed a cycle,
    the vector $\vec{v}$ mus also completed one cycle as it is the same as
    where it started after $t$ seconds. Similar to the derivation of $v$, the
    vector $\vec{v}$ can be thought as a clock's hand, sweeping across the circle with
    angular velocity $\omega$. So that the person on the end point of the
    vector or the clock's hand is experiancing, the velocity of the velocity
    vector $a$ aka Centripital acceleration.
    \[a = \omega v\]
    Or using that the circumference of the circle is equal in both expressions:
    \[a t  = \omega v t\]
    \[a = \omega t\]
    Since $v = \omega r $, the commonly usable formulas are:
        \begin{gather*}
            a = \omega ^2 r\\
            a = \frac{v^2}{r}\\
        \end{gather*}
    \section{Centripital Force}\label{sec:centripital-force}
    Centripital force can be in many different forms:
    \begin {enumerate}
        \item gravitational force such as the Sun and Earth, and the Earth
        and Moon.
        \item String tension swinging a rope in a circle.
        \item normal force in the space station is rotating, and the person
        inside of the car when car is turning.
        \item Spring force when a spring is rotating.
        \item static friction when the car is turing. It is static because no
        heat is generated as the result of turning the car.
    \end{enumerate}
    The centripital force is the net result of the other forces on the object,
    it alone is not a force. It is simply the force that keeps an
    object in circular motion.
    \[F_c = m a_c = m \omega ^2 r = m\frac{v^2}{r}\]

    \section{Work done by Centripital Force}
        When the object is has no tangential acceleration, at that short
        period of time, it can be thought that the object is in uniform
        circular motion.
        This is because the kinetic energy of the object at all points are:
            \[KE = \frac{1}{2} m v^2\]
        Since the kinetic energy is constant, the work done by the
        centripital force is Zero.
    \section{Tangential Acceleration}
        When the object is not in uniform circular motion, the object is
        experiencing a tangential acceleration. The tangential acceleration
        is the rate of change of the magnitude of the velocity. The
        acceleration is in the direction of the velocity vector, and it is
        the result of the net force on the object.
        \[a_t = \frac{dv}{dt}\]
        \[a_t = \frac{d}{dt}(\omega r) = r \frac{d\omega}{dt}\]
        \[a_t = r \alpha\]
        Where $\alpha$ is the angular acceleration, the rate of change of the
        angular velocity.
    \section{Moment of Inertia}
    When calculating the kinetic energy of an object in circular motion,
    assuming it is a point mass, the formula is:
        \[KE = \frac{1}{2}mv^2\]
    Since $v = \omega r$, the formula can be rewritten as:
        \[KE = \frac{1}{2}m(\omega r)^2 = \frac{1}{2}m r^2 \omega^2\]
    The term $mr^2$ is called the moment of inertia($I$) of the object which
    measures how object resists to rotate, just like how mass measures
    resistence an object is to move linearly.


\end{document}
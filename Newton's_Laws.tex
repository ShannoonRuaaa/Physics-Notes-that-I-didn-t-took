\documentclass[]{article}

%opening
\title{Newton's Laws}
\author{}

\begin{document}

\maketitle

\section{Newton's Laws}
	\subsection{The 3 Laws}
	\begin{enumerate}
		\item[Law 1:] Things have mass, and they sit still when net force is 0. 
		\item[Law 2:] $\Sigma F=ma$
		\item[Law 3:] Push $+$ pull $=$ $0$
	\end{enumerate}
	\subsection{Law 1}
		Mass is an abstract concept, artificially defined to measure how hard it is to move an object. Therefore, under a different scenario, it is possible to use another definition of mass. For now, we will look at every object as a point mass.
		
		A point mass is an object without a shape.
	\subsection{Law 2}
		This is the only useful formula, with variations like:
		$$a= \frac{\Sigma F}{m}$$
		$$m = \frac{\Sigma F} {a}$$
		Use this formula at your convenience.
	\subsection{Law 3}
		Just remember to draw a pair of arrows when drawing free body diagrams. This concept will be especially useful while understanding conservation of momentum.
	
	\section{Question:}
	\begin{enumerate}
		\item If all objects accelerate towards Earth at the same rate, why does it hurt more to be hit by an iron ball than to be hit by a football?
	\end{enumerate}
\end{document}

\documentclass[]{article}

%opening
\title{Newton's laws}
\author{}

\begin{document}

\maketitle

\section{Newton's Laws}
	\subsection{The 3 Laws}
	\begin{enumerate}
		\item[Law 1:] Things has mass, they sit still when net force is 0. 
		\item[Law 2:] $\Sigma F=ma$
		\item[Law 3:] Push $+$ pull $=$ $0$
	\end{enumerate}
	\subsection{Law 1}
		Mass is an abstract concept, it is an artificially defined to measure how hard it is to move an object. Therefore it is possible to use another definition of mass under different scenarios. We will look at every object as a point mass for now.
		
		Point mass is an object without a shape
	\subsection{Law 2}
		The only useful formula, it also has variations like:
		$$a= \frac{\Sigma F}{m}$$
		$$m = \frac{\Sigma F} {a}$$
		Use the formula at your convenience.
	\subsection{Law 3}
		Just remember to draw a pair of arrows while drawing free body diagrams. However this concept will be useful while understanding conservation of momentum.
	
	\section{Question:}
	\begin{enumerate}
		\item If all object accelerate towards the Earth at the same rate, why does it hurt more to be hit by an iron ball than to be hit by a football?
	\end{enumerate}
\end{document}

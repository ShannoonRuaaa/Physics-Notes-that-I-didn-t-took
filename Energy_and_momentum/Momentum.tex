\documentclass[]{article}
\usepackage{amsmath}
%opening
\title{Energy and Momentum}
\author{}

\begin{document}

\maketitle
\section{Work and Energy}\label{sec:work-and-energy}
The two words are often used in the same context, each coming from a unique understanding describing a similar situation.
These concepts are often used when the problem gets too complicated to come up with an explicit formula for modeling the system.

\subsection{Definition}\label{subsec:definition}
\[\text{Work done by } F = \vec{F} \cdot \vec{distance}\]
	 In work done by conservative forces, the work done is recoverable.
	It can also be thought of as the potential energy that is the consequence of the work done by such forces.
	When work is done by non-conservative forces, it is not recoverable and can be thought to be turned into heat. 

	Because work being done by conservative forces is turned into potential energies, it can be thought that the total energy of the system remains the same when work is done by conservative forces.
	The total energy of the system can only be changed by non-conservative or external forces.
	
	In summary,
	\begin{gather*}
	    \Delta E = 0 \text{ when the forces are all conservative}\\
	    \Delta E = W_{NC} \text{ when non-conservative forces are present}\\
	\end{gather*}
\section{Formulas}\label{sec:formulas}

\subsection{Friction}\label{subsec:friction}
When two surfaces are rubbing against each other, they will heat up very easily.
Therefore, the work done by friction is transferred into heat energy, making the friction force a non-conservative force.
As a non-conservative force, work done by friction is dependent on the path it takes.
	\[W_{fr} = F_{fr}\cdot \Delta x = \mu F_N \cdot \Delta x\]

	\subsection{Kinetic Energy}\label{subsec:kinetic-energy}
\begin{gather*}
	    \Delta KE = m \frac{v_f^2 - v^2_i}{2}\\
	    KE = m\frac{v^2}{2}\\
	\end{gather*}

	We will only derive the $1$D scenario for the sake of understanding. 
	
	Because in classical physics, the mass of an object doesn't change, thus $m$ remains constant.
	$F$ is a constant force.
	
	We will use $W_F$ to denote the work done by $F$.
	
	$\Delta x$ will represent the change in distance.
	\begin{gather*}
	    W_F = F \cdot \Delta x\\
	    W_F = m (a \cdot \Delta x)\\
	\end{gather*}
	Recall in kinematics, $v_f^2 = v^2_i + 2a\Delta x$
	\begin{gather*}
	    v_f^2 - v^2_i = 2a\Delta x\\
	    \frac{v_f^2 - v^2_i}{2} = a\Delta x\\
	\end{gather*}
	So replace $a\Delta x$ into the original expression.
	\[W_F = m \frac{v_f^2 - v^2_i}{2}\]
	Assuming that a particle traveled from $A$ to $B$ with constant acceleration $a_1$ and from $B$ to $C$ with acceleration $a_2$, to calculate the work done from $A$ to $C$ we can add the work done in the first portion and the second portion.
	
	\begin{gather*}
	    W_{AB} = m\frac{v_A^2 -v_B^2}{2}\\
	    W_{BC} = m\frac{v_B^2 -v_C^2}{2}\\
	    W_{AC} = W_{AB} + W_{BC} = m\frac{v_A^2 -v_C^2}{2}\\
	\end{gather*}
	So the original expression holds when $a$ is not a constant term, or in other words when $F$ is not a constant force. 
	\subsection{Potential Energy}\label{subsec:potential-energy}

\subsubsection{Spring}
			By Hooke's law, 
			\[F = -k (x-x_0) = -k\Delta x\]
			Since it is a linear graph, we can use the formula for area of an triangle and get:
				\[PE = \frac{k\Delta x^2}{21}\]

		
\section{Momentum}\label{sec:momentum}
\[L = mv\]
Assuming you are at an inertial frame of reference, an object's momentum relative to you is $mv$.
A change in an object's momentum is:
\[\Delta L = m \Delta v\]
A change in momentum must be caused by an outside force because of the second equation in kinematics ($\Delta v = \Sigma a\Delta t$).

\begin{gather*}
    \Delta L = m\Delta v\\
    \Delta L = m\Sigma a \Delta t\\
    \Delta L = \Sigma m a \Delta t\\
    \Delta L = \Sigma F \Delta t\\
\end{gather*}
Therefore, momentum is an alternate way of representing Newton's third law.
Without outside forces, F will be $0$ (Newton's Third law) making the change of momentum $0$.
 

\section{Collisions}\label{sec:collisions}

\subsection{Elastic}\label{subsec:elastic}
means that two objects does NOT stick together after the collision.
Perfect elastic collisions are those in which both kinetic energy and momentum are conserved.
For a collision between two bodies labeled 1 and 2, the following two equations describe an elastic collision:

\[E = \frac{1}{2} m_1 |\mathbf{v}_{1i}|^2 + \frac{1}{2} m_2 |\mathbf{v}_{2i}|^2 = \frac{1}{2} m_1 |\mathbf{v}_{1f}|^2 + \frac{1}{2} m_2 |\mathbf{v}_{2f}|^2\]

\begin{enumerate}
	\item $|\mathbf{v}_{1i}|$ and $|\mathbf{v}_{2i}|$ are the magnitudes of the initial velocities of bodies 1 and 2,

	\item $|\mathbf{v}_{1f}|$ and $|\mathbf{v}_{2f}|$ are the magnitudes of the final velocities of bodies 1 and 2 after collision.

\end{enumerate}
\[L = m_1 \mathbf{v}_{1i} + m_2 \mathbf{v}\_{2i} = m_1 \mathbf{v}_{1f} + m_2 \mathbf{v}_{2f}\]

\begin{enumerate}
\item $m_1$ and $m_2$ are the masses of bodies 1 and 2 respectively,

\item $v_{1i}$ and $v_{2i}$ are the initial velocities of bodies 1 and 2,

\item $v_{1f}$ and $v_{2f}$ are the final velocities of bodies 1 and 2 after collision.
  
\end{enumerate}
\subsubsection{Inelastic} means that the two objects stick together after the collision.
In an inelastic collision, momentum is conserved but kinetic energy is not.
Instead, some of the kinetic energy is converted into potential energy, holding the body together (converted into heat,
sound, or movement are also acceptable explanations).
\[m_1 \mathbf{v}_{1i} + m_2 \mathbf{v}_{2i} = (m_1 +m_2)\mathbf{v}_{f}\]
When a body is split into two or more separate parts, it is also a type of inelastic collision called a reverse inelastic collision.
\[(m_1 +m_2)\mathbf{v}_{i} = m_1 \mathbf{v}_{1f} + m_2 \mathbf{v}_{2f} \]

To understand why energy is not conserved conceptually, we can use a reverse inelastic collision as an example.
When the observer is traveling at the same speed as $\mathbf{v}_{i}$, the object will remain static to the observer.
At some point in time, the observer will see the parts of the object traveling at a speed of $\mathbf{v}_{1f} -\mathbf{v}_{i}$ and  $\mathbf{v}_{2f} -\mathbf{v}_{i}$ away from the observer.
It can be seen that the kinetic energy went from $0$ relative observer to $\frac{1}{2} m_1(\mathbf{v}_{1f} -\mathbf{v}_{i})^2 + \frac{1}{2} m_2 (\mathbf{v}_{2f} -\mathbf{v}_{i})^2$.
Therefore, the energy is not conserved.
\[\frac{1}{2} m_1 |\mathbf{v}_{1i}|^2 + \frac{1}{2} m_2 |\mathbf{v}_{2i}|^2 \neq \frac{1}{2} m_1 |\mathbf{v}_{1f}|^2 + \frac{1}{2} m_2 |\mathbf{v}_{2f}|^2\]


\section{Question}\label{sec:question}
\begin{enumerate}
	\item Emma carried her pet goldfish to walk from her parents' house to her grandparents' house, walking 2 miles to get to her destination.
	Why did she not do any work on the goldfish even though she felt exhausted after the walk?
	(Do not use definition, answer it conceptually, and with full understanding of the concept)
\end{enumerate}


\end{document}

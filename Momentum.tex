\documentclass[]{article}
\usepackage{amsmath}
%opening
\title{Momentum}
\author{}

\begin{document}

\maketitle
\section{Work and Energy}
The two words can be used interchangeably, each came from an unique understanding, while they are describing the same situation. These concepts are often used when the problem gets too complicated to come up with an explicit formula for  

\subsection{Definition}
	$$\text{Work done by } F = \vec{F} \cdot \vec{distance}$$
	In work done by conservative forces, the work done is recoverable. Or it can be thought as the potential energy that is the consequence of the work done by such forces.
	When work is done by non-conservative force, it is not recoverable and can be thought to turned into heat. 

	Because work being done by conservative forces are turned into potential energies, it can be thought as the energy of the system remain the same when work is done by conservative forces. The energy of the system can only be changed by non-conservative or external forces.
	
	In summary,
	$$\Delta E = 0 \text{ When the forces are all conservative}$$
	$$\Delta E = W_{NC} \text{ When non-conservative forces are present}$$ 
\section{Formulas}	
	\subsection{Kinetic Energy}
	$$\Delta KE = m \frac{v_f^2 - v^2_i}{2}$$
	$$KE = m\frac{v^2}{2}$$

	We will only derive the $1$D scenario for the sake of understanding. 
	
	Because in classical physics, mass of an object doesn't change, $m$ remain constant.
	$F$ is a constant force.
	
	We will use $W_F$ to denote the work done by F.
	
	$\Delta x$ will represent  the change in distance.
	$$W_F = F \cdot \Delta x$$
	$$W_F = m (a \cdot \Delta x)$$
	Recall in kinematics, $v_f^2 = v^2_i + 2a\Delta x$
	$$v_f^2 - v^2_i = 2a\Delta x$$
	
	$$\frac{v_f^2 - v^2_i}{2} = a\Delta x$$
	So replace $a\Delta x$ into the original expression.
	$$W_F = m \frac{v_f^2 - v^2_i}{2}$$
	
	\subsection{Potential Energy}
		\subsubsection{Spring}
			By Hooke's law, 
			$$F = -k (x-x_0)$$
			Since 	
\section{Momentum}
Assuming you are at an inertial frame of reference, an object's momentum relative to you is $mv$. A change in an object's momentum is:
$$\Delta L = m \Delta v$$
A change in momentum must be caused by an outside force because of the second equation in kinematics ($\Delta v = a\Delta t$).

So when the acceleration is constant
\section{Question}
\begin{enumerate}
	\item Emma carried her pet gold fish to walk from her parents' house to her grandparents' house, she walked for 2 miles to get to her destination. Why did she not do any work on the gold fish? (Do not use definition, answer it conceptually, and with full understanding of the concept)
\end{enumerate}
\section{Conservation Laws}

\end{document}

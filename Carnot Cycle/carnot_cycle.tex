%! Author = shannon
%! Date = 10/19/24

% Preamble
\documentclass[11pt]{article}

% Packages
\usepackage{amsmath}
\title{Carnot Cycle}
\author{}
% Document
\begin{document}
\maketitle
\section{Carnot Cycle}
    The Carnot cycle is a theoretical thermodynamic cycle that is the most efficient cycle possible for converting a given amount of thermal energy into work or conversely, using work to pump heat. The cycle consists of four reversible processes:
    \begin{enumerate}
        \item Isothermal Expansion: The working substance absorbs heat from a high-temperature reservoir at constant temperature $T_H$ and expands, doing work on the surroundings.
        \item Adiabatic Expansion: The working substance expands further, doing work on the surroundings without exchanging heat with the surroundings.
        \item Isothermal Compression: The working substance is compressed at constant temperature $T_L$, rejecting heat to a low-temperature reservoir.
        \item Adiabatic Compression: The working substance is compressed further, rejecting heat to the surroundings without exchanging heat with the surroundings.
    \end{enumerate}
    The efficiency of heat engine is given by:
    \[\eta = 1- \frac{Q_l}{Q_i}\]
    Where $\eta$ is the efficiency of the engine, $Q_l$ is the heat rejected
to the low-temperature reservoir, and $Q_i$ is the heat absorbed from the
high-temperature reservoir.
    In the Carnot cycle,the only heat transfer occurs at the two isothermal,
the heat transfer in the adiabatic processes is zero.
\begin{gather*}
    Q_i = nRT_H \ln(V_B/_A)\\
    Q_l = nRT_L \ln(V_C/_D)\\
\end{gather*}
Combine the two equations into the efficiency expression, we get:
\[\eta = 1 - \frac{nRT_L \ln(V_C/_D)}{nRT_H \ln(V_B/_A)}\]
Because adiabatic processes, we are given expressions (proof in appendix):
\[V_B/V_A = V_C/V_D\]
    The efficiency of the Carnot cycle is given by:
    \[\eta = 1 - \frac{T_L}{T_H}\]
    where $\eta$ is the efficiency of the cycle, $T_L$ is the temperature of the low-temperature reservoir, and $T_H$ is the temperature of the high-temperature reservoir.
\Appendix
\section{Proof of Adiabatic Processes}\label{sec:proof-of-adiabatic-processes}
    The adiabatic process have the following expression:
    \[PV^\gamma = \text{constant}\]
    Where $P$ is the pressure, $V$ is the volume, and $\gamma$ is the adiabatic index.
    For each point in the Carnot cycle, we have the following expressions:
    \begin{align*}
        P_B V_B^\gamma = c_1\\
        P_C V_C^\gamma = c_1\\
        P_A V_A^\gamma = c_2\\
        P_D V_D^\gamma = c_2\\
    \end{align*}
    Where $c_1$ and $c_2$ are constants.
    \[
        \frac{P_A V_A^\gamma}{P_B V_B^\gamma} = \frac{P_D V_D^\gamma}{P_C V_C^\gamma} = \frac{c_1}{c_2}
    \]
    The isothermal processes have the following expressions:
    \begin{align*}
        P_A V_A = P_B V_B\\
        P_C V_C = P_D V_D\\
    \end{align*}
    Therefore, we have:
    \[
        \frac{P_A}{P_B} = \frac{V_B}{V_A}
    \]
    \[\frac{V_C}{V_D} = \frac{P_D}{P_C}\]
    Combining the Isothermal and Adiabatic processes, we get:
\[
    \frac{V_A^{\gamma-1}}{V_B^{\gamma-1}} = \frac{ V_D^{\gamma-1}}{ V_C^{\gamma-1}}= \frac{c_1}{c_2}
\]
\[V_B/V_A = V_C/V_D\]
\end{document}
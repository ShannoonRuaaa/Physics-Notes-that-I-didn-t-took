%! Author = shann
%! Date = 6/16/2024

% Preamble
\documentclass[11pt]{article}

% Packages
\usepackage{amsmath}

% Document
\begin{document}
\section{Simple Harmonic motion}
    Simple harmonic motion is a type of periodic motion where the restoring force is directly proportional to the displacement.
    The motion is sinusoidal in time and demonstrates a single frequency. The equation of motion is:
    \[x(t) = A \cos(\omega t + \phi)\]
    where:
    \begin{itemize}
        \item $x(t)$ is the position of the object at time $t$
        \item $A$ is the amplitude of the motion
        \item $\omega$ is the angular frequency of the motion
        \item $\phi$ is the phase angle
    \end{itemize}
    The angular frequency $\omega$ is related to the period $T$ of the motion by:
    \[\omega = \frac{2\pi}{T}\]
    The velocity and acceleration of the object in simple harmonic motion are given by:
    \[v(t) = -A\omega \sin(\omega t + \phi)\]
    \[a(t) = -A\omega^2 \cos(\omega t + \phi)\]
    The velocity is out of phase with the position by $\frac{\pi}{2}$, and the acceleration is out of phase with the position by $\pi$.
\section{Spring Mass System}
    A common example of simple harmonic motion is the spring-mass system, where a mass $m$ is attached to a spring with spring constant $k$.
    The equation of motion for the spring-mass system is:
    \[m\frac{d^2x}{dt^2} = -kx\]
    This equation can be solved to obtain the position of the mass as a function of time.
    The angular frequency $\omega$ of the spring-mass system is given by:
    \[\omega = \sqrt{\frac{k}{m}}\]
    The period $T$ of the motion is given by:
    \[T = \frac{2\pi}{\omega} = 2\pi\sqrt{\frac{m}{k}}\]
    The velocity and acceleration of the mass in the spring-mass system are given by:
    \[v(t) = -A\omega \sin(\omega t + \phi)\]
    \[a(t) = -A\omega^2 \cos(\omega t + \phi)\]
\section{Pendulums}
    Another not so standard example of simple harmonic motion is the
    pendulum. Although the motion of a pendulum is not perfectly simple harmonic,
    it can be approximated as such for small angles.
    Using Work Energy Theorem, the total energy of the system is conserved.
    Therefore the sum of the kinetic and potential energy is constant.
    \[E = KE + PE = \frac{1}{2} m v^2 + mgh = \text{constant}\]
    In this context, the variable $h$ can be represented as $l - l\cos\theta$.
    Velocity can be represented as $v = l\omega$.
    Therefore, the total energy can be represented as:
    \[E = \frac{1}{2} m l^2 \omega^2 + mgl(1 - \cos\theta)\]
    Since the change in energy is zero, the derivative of the energy is zero.
    \[\frac{dE}{dt} = 0\]
    \[\frac{d}{dt} \left(\frac{1}{2} m l^2 \omega^2 + mgl(1 - \cos\theta)\right) = 0\]
    \[m l^2 \omega \alpha - mgl\sin\theta \frac{d\theta}{dt} = 0\]
    \[l \alpha - g\sin\theta = 0\]
    Since$\sin \theta$ is very close to $\theta$ for small angles, the equation can be simplified to:
    \[\alpha + \frac{g}{l} \theta = 0\]
    The angular frequency $\omega$ of the pendulum is given by:
    \[\omega = \sqrt{\frac{g}{l}}\]
    The period $T$ of the motion is given by:
    \[T = 2\pi\sqrt{\frac{l}{g}}\]
    When finding the velocity of the pendulum, it is most common to use work
    enregy theorem. The velocity of the pendulum is given by:
    \[E_0 = mgl(1-\cos \theta _0) =  \frac{1}{2} m v^2 + mgl(1 - \cos\theta)\]
    \[v = \sqrt{2gl( \cos\theta -\cos \theta_0 )}\]

\section{Resonance Frequency}
Assuming there is a force that hits the spring mass system at a frequency $f$
. It would be the most optimal to hit the spring mass system when the mass
is at its largest velocity and the force is the same direction with the
velocity vector. This is because the force will add to the velocity of the
mass, increasing the kinetic energy of the system as well as the amplitude
for each period. As time goes on, the amplitude can approach to infinity in
the ideal case. Based on the observation the
frequency $f$ must be the same as the frequency of the spring mass system.
However, it doesn't have to be at the same timing, as the force will slowly
overcome the damping phase and be in phase with the system.
\end{document}
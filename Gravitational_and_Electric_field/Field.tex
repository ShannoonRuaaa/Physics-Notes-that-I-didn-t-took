\documentclass[]{article}

%opening
\title{Gravitational and Electrostatic}
\author{}

\begin{document}
	\include{physics}
	\include{esint}
	\include{amsmath}


\maketitle


\section{Formula}
The strength of gravity from the Earth surface is approximately the same, therefore in most problems we approximate $g=9.8\frac{m}{s^2}$. However, when as the distance from center of Earth increases, $g$ no longer holds to be a constant, but rather a function of the distance from center of Earth, $r$.
$$F = G \frac{m_1m_2}{r^2}$$
At the same time, electro static forces generated by a point charge also have a similar formula.
$$F = k\frac{q_1q_2}{r^2} = \frac{q_1q_2}{4\pi\epsilon_0 r^2}$$
In both formulas, constant $k$ and $g$ are scalars that can be interpreted as strength of the field. In Coulomb's law, the constant $k$ can change based on medium of the material the charged objects are immersed in, therefore it is written in an alternative $\frac{1}{4\pi\epsilon_0}$, where $\epsilon_0$ is the permittivity of free space. 

\section{Symmetry Argument}
When the gravitational/electric force is generated by a point, objects that are $r$ units away will all experience force with the same magnitude and different directions -- This is the symmetry argument, they are the same.   

When dealing with g/e force generated by an infinitely long rod, a concentric cylinder enclosing the rod is a set (group) of points that experience force with same magnitude and different directions.

If it is an infinitely large plane that is producing the e/g force, two parallel planes will experience forces with same magnitude but different directions.

\section{Green's Theorem and Gaussian Surface }
$$\oint_c F \cdot \,dl = \iint_R curl(F) \,dA  $$

\section{Why $\frac{1}{r^2}$}

\end{document}

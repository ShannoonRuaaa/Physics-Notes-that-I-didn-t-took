\documentclass[]{article}
\usepackage{amsmath}

%opening
\title{Newton's Laws}
\author{}

\begin{document}

\maketitle

\section{Newton's Laws}\label{sec:newton's-laws}

\subsection{The 3 Laws}\label{subsec:the-3-laws}
\begin{enumerate}
		\item[Law 1:] Things have mass, and they sit still when net force is 0. 
		\item[Law 2:] $\Sigma F=ma$
		\item[Law 3:] Push $+$ pull $=$ $0$
	\end{enumerate}
	\subsection{Law 1}
		Mass is an abstract concept, artificially defined to measure how hard it is to move an object.
		Therefore, under a different scenario, it is possible to use another definition of mass.
		For now, we will look at every object as a point mass.
		
		A point mass is an object without a shape.
	\subsection{Law 2}
		This is the only useful formula, with variations like:
		\begin{gather*}
		    a_{net}= \frac{\Sigma F}{m}\\
		    m = \frac{\Sigma F} {a_{net}}\\
		\end{gather*}
		Use this formula at your convenience.
	\subsection{Law 3}
		Just remember to draw a pair of arrows when drawing free body diagrams.
		This concept will be especially useful while understanding conservation of momentum.
	\section{Different Forces:}\label{sec:different-forces:}
	\begin{enumerate}
		\item Normal Force: The force that prevents objects from falling
		through each other, a responsive force when surfaces push against
		each other.
		\[F_N\]
		\item Tension: The force transmitted through a string, rope, cable or
		similar one-dimensional continuous object.
		\[T\]
		\item Friction: The force that prevents objects from sliding past
		each other, a resistive force that always acts in the opposite
		direction to the motion.
		\[f = \mu F_N\]
		It is linearly proportional to the normal force, $\mu$ is called the
		coefficient of friction. $\mu$ can be dependent on the
		material of two surfaces, it can also change when the object
		is in static or kinetic.
		Usually $\mu_k < \mu_s$.
		\item Gravity: The force that pulls objects towards each other.
		When
		the object is close to the surface of the Earth, the force is:
		\[F = mg\]
		However, when the object is far from the surface of the Earth, the
		strength of the force falls off by square of the distance from the
		center of the Earth.
		\begin{gather*}
		    F = -\frac{Gm_1m_2}{r^2}\\
		    a_g = -\frac{Gm_1}{r^2}\\
		\end{gather*}
		$G$ is the gravitational constant that can be interpreted as the
		strength of the force between two objects. Notice acceleration due to
		gravity is linearly proportional to the mass of the object.

		\item Spring Force: The force that prevents objects from stretching
		or compressing a spring. It is linearly proportional to the distance
		of compression or stretching. The negative sign indicates that the
		force is always in the opposite direction to the displacement.
		\[F = -k\Delta x\]
	\end{enumerate}
	\section{Question:}\label{sec:question:}
	\begin{enumerate}
		\item If all objects accelerate towards Earth at the same rate, why does it hurt more to be hit by an iron ball than to be hit by a football?
	\end{enumerate}
\end{document}

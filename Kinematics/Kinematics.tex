\documentclass[]{scrreprt}

\usepackage{amsmath}
% Title Page
\title{}
\author{}


\begin{document}
\maketitle

\begin{abstract}
	
\end{abstract}
\begin{section} {Kinematics}
	\subsection{Key terms:}
		\begin{enumerate}
			\item[Acceleration:]
				a change in velocity over time
			
			\item [Velocity:] 
				a change in position over time, or the result of accumulated acceleration.
			
			\item [Position:]
			 	where the particle is at, or the result of accumulated velocity.
		\end{enumerate}
		
	\subsection{Basic Formulas:}
		\begin{enumerate}
			\item $x(t) = \frac{1}{2}at^2 + v_0t+x_0$
			\item $v(t)=at+v_0$
			\item $a(t) = a_0$
		\end{enumerate}	
	
	\subsection{Change Formulas:} 
		Defining $\Delta A = A - A_0$, meaning that the change in $A$ is the difference of the variable $A$ and the initial $A$ when $t = 0$ (or just simply $A_0$).
			\begin{enumerate}
				\item $\Delta x(t) = \frac{1}{2}at^2 + v_0t$
				\item $\Delta v(t)=at$
				\item $\Delta a(t) = 0$	
			\end{enumerate}
		Notice how only difference is that the constant terms disappeared because it is simplified to the term $\Delta$ which is the change of that variable. 
	
		\subsection{Extended formula:}
		$$v^2 = v_0^2+2a\Delta x$$
		$$x=x_0+\frac{(v_0+v)}{2}t$$
	
	\begin{subsection}{Understand the concept}
		To understand the relationship quantitatively, we will show the relation between $v$, $a$, and $x$ and derive their relation from the base.
		Assuming the only formula given is:
		$$a(t) = a_0$$
		we will start from $a$ to $v$:
		the area under the graph $a$ vs $t$ as shown below:
		%insert graph
		the area is strictly a rectangle, therefore the term $at$ represents the accumulation of acceleration over the $t$. And with satisfactory, we have the formula 2, adding the constant term, we have the second formula.
		
		from $v$ to $a$:
		$v(t) = =at+v_0$
		With keen observation, the shape is simply a trapezoid. Divide it into a rectangle and triangle, 
		%insert graph
		Area of rectangle = $v_0t$
		
		Area of triangle = $\frac{1}{2} a t^2$
		
		sum it together gives formula 1, adding the constant term, we have the first formula.
		
		Notice the rectangle from the divided term is from its initial velocity, and the triangle is responsible for the acceleration. If the acceleration is $0$, the triangle will have $0$ area and the $v$ vs $t$ graph is a constant line again. If initial velocity is $0$, the graph will look like a triangle, and the position equation is $x(t) = 1/2 a t^2$
				
		Now we will start from position formula and work down the road to show that in a parabolic graph, acceleration is constant.
		Given $x(t) = \frac{1}{2}at^2 + v_0t+x_0$ 
		If we want to find the velocity at specific point $t$, we will find the change in position from $t-\delta$ to $t+\delta$ over a period of $2\delta$.
		 $$v(t)\approx \frac{x(t+\delta)-x(t-\delta)}{2\delta}$$
		 $$v(t)\approx \frac{(\frac{1}{2}a(t+\delta)^2 + v_0(t+\delta) + x_0)-
		 					 (\frac{1}{2}a(t-\delta)^2 + v_0(t-\delta)+x_0)}{2\delta}$$
	 	Combine like terms,
	 	$$v(t)\approx \frac{[\frac{1}{2}a(t+\delta)^2-
	 		\frac{1}{2}a(t-\delta)^2] + [v_0(t+\delta)-v_0(t-\delta)]+[x_0-x_0])}{2\delta}$$
		 $$v(t) \approx \frac{[\frac{1}{2}a((t+\delta)+(t-\delta))\cdot((t+\delta)-(t-\delta))}{2\delta} + v_0$$
		
		$$v(t) \approx \frac{[\frac{1}{2}a((2t)\cdot(2\delta)}{2\delta} + v_0$$ 	
		$$v(t)\approx at+v_0$$
		So the value of delta doesn't matter in the calculation, and can be equal to any number that is not $0$. So if we allow it to be $10^{-99999999}$ the $v(t)$ is the same as $\delta = 10^{9999999}$.
		$$v(t) = at+v_0$$
		and since this graph is linear,
		$a(t) = \frac{\delta v}{t} =a$		
		We will gain a deeper understanding of the more complicated formula in Work Energy, but to conclude,
		$$x = Sum (v) + x_0$$
		$$v = \frac{\delta x}{\delta t} = Sum (a) + v_0$$
		$$a = \frac{\delta v}{\delta t} = constant$$ 	
\end{subsection} 
\subsection{Kinematics in vectors}
If the question require multiple dimensions, the formulas in scalar form works as follows.
$$\vec{a} = 
\begin{bmatrix}
	a_1\\
	a_2
\end{bmatrix}$$
$$\vec{v} = 
\begin{bmatrix}
	a_1 \\
	a_2 
\end{bmatrix}
t+
\begin{bmatrix}
	v_1\\
	v_2
\end{bmatrix}
$$
$$\vec{x} = 
\frac{1}{2}
\begin{bmatrix}
	a_1 \\
	a_2 
\end{bmatrix}
t^2+
\begin{bmatrix}
	v_1\\
	v_2
\end{bmatrix}
t+
\begin{bmatrix}
	x_1\\
	x_2
\end{bmatrix}
$$

\end{section}

\subsection{problem}
1. Show that $\frac{v+v_0}{2x} =\frac{v-v_0}{a}$ using either graph or equation and then prove the extended formula.

2. Without using the extended formula, how fast will a steel ball be falling from a window that is $10$ m from the ground? How long will it take? And how about a wood ball?


\end{document}          
